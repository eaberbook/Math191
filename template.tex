%%%%%%%%%%%%%%%%%%%%%%%%%%%%%%%%%%%%%%%%%%%%%%%%%%%%%%%%%%%%%%%%
%  Template for creating scribe notes for MAT 308
%
%  Fill in your name, lecture number, lecture date and body
%  of scribe notes.
%%%%%%%%%%%%%%%%%%%%%%%%%%%%%%%%%%%%%%%%%%%%%%%%%%%%%%%%%%%%%%%%

\documentclass[11pt]{article}

% \setlength{\topmargin}{9pt}
% \setlength{\textheight}{9in}
% \setlength{\headheight}{2pt}
% \setlength{\headsep}{4pt}
% \setlength{\oddsidemargin}{0.25in}
% \setlength{\textwidth}{6in}
% \pagestyle{plain}


\usepackage{amsmath,amssymb,amsrefs}
\usepackage{amscd}
\usepackage{latexsym}
\usepackage{graphics}


\setlength{\oddsidemargin}{0in}
\setlength{\evensidemargin}{0in}
\addtolength{\topmargin}{-1in}
\setlength{\textwidth}{6.5in}
\setlength{\textheight}{8in}

\usepackage{algorithm}
\usepackage{algorithmic}

% \setlength{\topmargin}{0in}
% \setlength{\headheight}{0in}
% \setlength{\headsep}{0in}
% \setlength{\textheight}{7.7in}
% \setlength{\textwidth}{6.5in}
% \setlength{\oddsidemargin}{0in}
% \setlength{\evensidemargin}{0in}
% \setlength{\parindent}{0.25in}
% \setlength{\parskip}{0.25in}

\usepackage{subfigure}
\usepackage{graphicx}

%
% this command enables to remove a whole part of the text 
% from the printout
% to use it just enter
% \remove{  
% before the text to be excluded and
% } 
% after the text
\newcommand{\remove}[1]{}

%
% The following macros are used to generate nice code for programs.
% See example on how to use it below
%

%%%%%%%%%%%%%%%%%%%%% program macros %%%%%%%%%%%%%%%%%


%%%%%%%%%%%%%%%%%%%%% End of PROGRAM macros %%%%%%%%%%%%%%%%%



\newcommand{\lecture}[5]{
   \pagestyle{headings}
   \thispagestyle{plain}
   \newpage
%   \setcounter{chapter}{#1}
%   \setcounter{page}{#2}
%  \set\thechapter{#3}
   \noindent
   \begin{center}
   \framebox{
      \vbox{
    \hbox to 6.28in { {\bf MATH 191 Topics in Data Science: Algorithms and Math. Foundations
                        \hfill  - Fall 2015} }
       \vspace{4mm}
       \hbox to 6.28in { {\Large \hfill Lecture #1: #3  \hfill} }
       \vspace{2mm}
       \hbox to 6.28in { {\it Lecturer: #4 \hfill Scribe: #5} }
      }
   }
   \end{center}
   \markboth{Lecture #1: #3}{Lecture #1: #3}
   \vspace*{4mm}
}

%
% Use these macros for organizing sections of your notes.
% Each command takes two arguments: (1) the title of the section and and
% (2) a keyword for that section to appear in the index.  (See examples.)
% Please don't use \section, \subsection, and \subsubsection directly!
%

\newcommand{\topic}[2]{\section{#1} \index{#2} \markright{#1}}
\newcommand{\subtopic}[2]{\subsection{#1} \index{#2}}
\newcommand{\subsubtopic}[2]{\subsubsection{#1} \index{#2}}
 
%
% Convention for citations is first author's last name followed by other
% authors' last initials, followed by the year.  For example, to cite the
% seventh entry in the course bibliography, you would type: \cite{BurnsL80}
% (To avoid bibliography problems, for now we redefine the \cite command.)
%

\renewcommand{\cite}[1]{[#1]}

%
% These are just to make things a little easier:
%
\newcommand{\bi}{\begin{itemize}}
\newcommand{\ei}{\end{itemize}}
\newcommand{\be}{\begin{enumerate}}
\newcommand{\ee}{\end{enumerate}}
\newcommand{\blank}{\vspace{1ex}}   % generates a blank line in the output

%
% Use these for theorems, lemmas, proofs, etc.
%
\newtheorem{theorem}{Theorem}
\newtheorem{lemma}[theorem]{Lemma}
\newtheorem{claim}[theorem]{Claim}
\newtheorem{corollary}[theorem]{Corollary}
\newcommand{\qed}{\hfill $\Box$}
% \newenvironment{proof}{\par{\bf Proof:}}{\qed \par}
\newenvironment{proof}{{\em Proof:}}{\hfill\rule{2mm}{2mm}}

%
% Use the following for definitions.
% \bigdef is for definitions to be set off by themselves; \smalldef is for
% definitions given in the middle of a paragraph.
%
\newenvironment{dfn}{{\vspace*{1ex} \noindent \bf Definition }}{\vspace*{1ex}}
\newcommand{\bigdef}[2]{\index{#1}\begin{dfn} {\rm #2} \end{dfn}}
\newcommand{\smalldef}[1]{\index{#1} {\em #1}}
% **** IF YOU WANT TO DEFINE ADDITIONAL MACROS FOR YOURSELF, PUT THEM HERE:
% \usepackage{subfigure}
% \usepackage{graphicx}


\begin{document}

\lecture{1}{1}{September 25, 2015}{Mihai Cucuringu}{!Your name goes here!}


%%%%%%%%%%%%%%%%%%%%%%%%%%%%%%%%%%%%%%%%%%%%%%%%%%%%%%%%%%%%%%%%
%%%%%%%%%%%%%%%%%%%%%%%%%%%%%%%%%%%%%%%%%%%%%%%%%%%%%%%%%%%%%%%%
%%%%%%%%%%%%%%%%%%%%%%%%%%%%%%%%%%%%%%%%%%%%%%%%%%%%%%%%%%%%%%%%
%%                 BODY OF SCRIBE NOTES GOES HERE             %%
%%%%%%%%%%%%%%%%%%%%%%%%%%%%%%%%%%%%%%%%%%%%%%%%%%%%%%%%%%%%%%%%
%%%%%%%%%%%%%%%%%%%%%%%%%%%%%%%%%%%%%%%%%%%%%%%%%%%%%%%%%%%%%%%%
%%%%%%%%%%%%%%%%%%%%%%%%%%%%%%%%%%%%%%%%%%%%%%%%%%%%%%%%%%%%%%%%

\section{This starts a new section...}

\begin{lemma}
\label{L7:mylemma}   % label used to refer to the lemma later. (see below)
                        % IMPORTANT: to keep labels unique among lectures,
                        % please precede every label with the lecture number,
                        % as shown (this prefix would be used for lecture 7.
This is the first lemma of the lecture.
\end{lemma}

\begin{proof}
The proof is by induction on \ldots
\end{proof}

\begin{theorem}
This is the first theorem.
\end{theorem}

\begin{proof}
This is the proof of the first theorem theorem.
\end{proof}


\subsection{This starts a new subsection...}

Early days:

\begin{itemize}
 \item 1928: \textbf{Minimax Theorem} for 2-person zero-sum games (John von Neumann)
 \item 1944: \textbf{Theory of Games and Economic Behavior} (von Neumann and Morgenstern)
\end{itemize}

We are now ready for a major definition.

\bigdef{myword}{This is the definition of a {\em saddle point}:}

\bigdef{ComKno}
Some information $\theta$ is {\em common knowledge} if for any sequence of players $i,j, \ldots ,k$ the statement  ``i knows that j knows that \ldots that k knows'' is true ( Note that $i,j, \ldots k$ may repeat, 
and the chain of letters may be of any length).

\begin{corollary}
This is a corollary following from ...
\end{corollary}

Sometimes we define terms in the middle of a paragraph. This is a \smalldef{different term} being defined.

On to the next page..

\newpage  % it's not necessary to do this before a figure -- latex should
          % position the figures appropriately

The matrix in Table \ref{mtx:ex1} is an example of a zero-sum game that does not have a saddle-point.

\begin{center}
$
\begin{array}{r|lc} 
 & A & B \\ 
\hline
A & 2 & -3 \\ 
B & 0  & 2 \\ 
C & -5 & 10 
\end{array}
\label{mtx:ex1}
$\end{center}

Here is how you add figures:

\begin{figure}[h]
\begin{center}
\subfigure[Our textbook]{\includegraphics[width=0.2\columnwidth]{Textbook.png}}
\subfigure[Grid Graph]{\includegraphics[width=0.32\columnwidth]{GridGraph.png}}
\subfigure[Grid Graph adjacency matrix]{\includegraphics[width=0.32\columnwidth]{GridGraph_SPY.png}}
\end{center}
\caption{ More description here...}
\label{fig:Game}
\end{figure}


To reference a figure :  Figure \ref{fig:Game} shows the textbook for the course (left), and the famous game \textit{Prisoner's Dilemma} (right).

%%%%%%%%%%%%%%%%%%%%%%%%%%%%%%%%%%%%%%%%%%%%%%%%%%%%%%%%%%%%%%%%


\begin{algorithm}[h!]
\begin{algorithmic}[1]
\REQUIRE $G=(V,E)$ the graph of pairwise comparisons and $C$ the $n \times n $ matrix of pairwise comparisons (rank offsets), such that whenever $ij \in E(G)$ we have available a (perhaps noisy) comparison between players $i$ and $j$, either a cardinal comparison ($C_{ij} \in [-(n-1), (n-1)]$) or an ordinal comparison $C_{ij} = \pm 1$.

\STATE Map all rank offsets $C_{ij}$ to an angle $\Theta_{ij} \in [0, 2 \pi  \delta )$ with $\delta \in [0,1)$, using the transformation 
\begin{equation}
	C_{ij}  \mapsto \Theta_{ij} :=  2 \pi  \delta  \frac{ C_{ij}}{n-1} 
	\label{transfToCircle}
\end{equation}
We choose $\delta = \frac{1}{2}$, and hence $\Theta_{ij} :=  \pi  \frac{ C_{ij}}{n-1} $.

\STATE Solve the angular synchronization problem over SO(2) using the eigenvector method, and denote the recovered solution by 
$ \hat{r}_i =  e^{\imath \hat{\theta}_i} = \frac{v_1^R(i)}{|v_1^R(i)|} , \;\;\;\; i=1,2,\ldots, n$.
\STATE Extract the corresponding set of angles $\hat{\theta}_1,\ldots,\hat{\theta}_n \in [0,2\pi)$ from $\hat{r}_1, \ldots,\hat{r}_n$.

%\RETURN 
\STATE  Output as a final solution the ranking induced by the circular permutation $\sigma$.
% \FOR{$i=1,\ldots,N-1$} 
% \STATE For each edge $ij$ of the graph, pick uniformly at random one of the 2 available measurements, and let the resulting matrix of (sampled) pairwise measurements be denoted by $\Theta_t$
% \STATE Solve $\Theta_t$ via eigenvector synchronization, and denote its solution by $s_t$
% \STATE Reconstruct the resulting denoised measurement matrix $ \hat{\Theta}^{(t)}_{ij} = s_i - s_i,  \forall ij \in E(G)$
\end{algorithmic}
\caption{ Example of an Algorithm}
\label{Algo:listSync}
\end{algorithm}

\clearpage




An example of a table is shown in Table  \ref{tab:methodAbbrev}.


\begin{table}[tpb]
\begin{minipage}[b]{0.99\linewidth}
\begin{center}
\begin{tabular}{|c|l|l|}
\hline
 Acronym & Name & Section \\
\hline
SVD 		&  SVD  Ranking &  Sec. 1  \\
LS 		& Least Squares Ranking & Sec. 2 \\
SER 		& Serial-Ranking &  Sec. 3  \\
SER-GLM 		& Serial-Ranking in the GLM model &  Sec. 4 \\
RC 			& Rank-Centrality &  Sec. 5  \\
SYNC & Synchronization-Ranking via the spectral relaxation &  Sec. 6 \\
SYNC-SUP		& Synchronization-Ranking based on the Superiority Score (spectral relaxation) &  Sec. 7 \\
SYNC-SDP 	& Synchronization-Ranking via the SDP relaxation & Sec. 8  \\
\hline
\end{tabular}
\end{center}
\end{minipage} 
\caption{Names of the algorithms we compare, their acronyms, and respective Sections.}
\label{tab:methodAbbrev}
\end{table}

\vspace{20mm}

See equation \ref{myExpValue} for how you can define a piece-wise function

\begin{equation}
\mathbb{E}[X] = \left\{
	\begin{array}{rl}
\sum_{i \in \Omega} x_i \; p(x_i) & \text{ for discrete rv} 	\\
 &  \\
 \int_{-\infty}^{\infty} x \; p(x) dx & \text{ for continuous rv}	
     \end{array}
   \right.
\label{myExpValue}
\end{equation}



\end{document}
